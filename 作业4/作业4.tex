%!TEX program = pdflatex
\documentclass[UTF8]{ctexart}
\usepackage{amsmath}
\usepackage{amssymb}
\usepackage{yhmath}
\usepackage{esvect}
\usepackage{arcs}
\usepackage{geometry}
\usepackage{graphicx}
\usepackage{subfigure}
\usepackage{float}
\usepackage{MnSymbol}
\usepackage{diagbox}
\usepackage{multirow}
\usepackage{makecell}
\usepackage{longtable}
\usepackage[justification=centering]{caption}
\geometry{left=2.0cm,right=2.0cm,top=2cm,bottom=2cm}
\newcommand*{\dif}{\mathop{}\!\mathrm{d}}
\newcommand*{\C}{^{\circ}\mathrm{C}}
\pagestyle{plain}

\begin{document}

\title{计算物理作业4}
\author{姓名:张健霖\\学号: \texttt{1900011327}}
\maketitle


\section{随机数的应用}

本题在样本1024个点中选点为有放回的选取,故而三角形种类$(x_1,x_2,x_3)$
在满足$x_1\geq x_2\geq x_3\geq 0, x_2+x_3\geq x_1-1$的约束下,
共有1056种可能,但实际体系中出现三角形种类数会比这个数小。

\noindent\textbf{(a)}

详见Possibility-a.xlsx文件,其三角形计数从$(x_1,x_2,x_3)=(0,0,0)$开始,
文件中可以发现部分三角形没有出现,总共出现三角形种类数目为838种,现列出出现概率前21的三角形。

误差用如下标准差计算
\begin{equation}\label{1-a-1}
    s_P=\sqrt{\frac{\sum_{i=1}^{S_N}(P_i-\bar{P})}{S_N(S_N-1)}}
\end{equation}

其中$S_N$为样本数目

\begin{table}[H]
    \centering
    \caption{均匀取点出现概率前21的三角形}
    \begin{tabular}{|ccc|ccc|ccc|}
        \hline
        $(x_1,x_2,x_3)$ & $P_a/\%$ & $\frac{s_P}{P}$ & $(x_1,x_2,x_3)$ & $P_a/\%$ & $\frac{s_P}{P}$ & $(x_1,x_2,x_3)$ & $P_a/\%$ & $\frac{s_P}{P}$\\
        \hline
        (9, 8, 7)&0.692&0.0017&(17, 15, 9)&0.5221&0.002&(17, 14, 8)&0.4811&0.002\\
(10, 9, 8)&0.5685&0.0019&(17, 13, 9)&0.5182&0.002&(18, 13, 9)&0.4801&0.0021\\
(10, 9, 7)&0.5586&0.002&(17, 16, 7)&0.5173&0.0019&(9, 8, 6)&0.4754&0.002\\
(17, 16, 8)&0.5506&0.002&(17, 13, 10)&0.5136&0.0019&(16, 15, 9)&0.4701&0.002\\
(17, 15, 8)&0.5427&0.0022&(17, 12, 10)&0.5015&0.002&(11, 9, 8)&0.4675&0.0021\\
(17, 14, 9)&0.5361&0.0021&(17, 12, 11)&0.4884&0.002&(18, 14, 9)&0.4659&0.0021\\
(10, 8, 7)&0.525&0.0022&(18, 12, 10)&0.4879&0.002&(18, 13, 10)&0.4585&0.002\\
\hline
    \end{tabular}
\end{table}

\noindent\textbf{(b)}

详见Possibility-b.xlsx文件,其三角形计数从$(x_1,x_2,x_3)=(0,0,0)$开始,
文件中可以发现部分三角形没有出现,总共出现三角形种类数目为837种,现列出出现概率前21的三角形,误差计算同(a)

\begin{table}[H]
    \centering
    \caption{球内取点出现概率前20的三角形}
    \begin{tabular}{|ccc|ccc|ccc|}
        \hline
        $(x_1,x_2,x_3)$ & $P_b/\%$ & $\frac{s_P}{P}$ & $(x_1,x_2,x_3)$ & $P_b/\%$ & $\frac{s_P}{P}$ & $(x_1,x_2,x_3)$ & $P_b/\%$ & $\frac{s_P}{P}$\\
        \hline
        (9, 8, 7)&0.6841&0.0017&(17, 15, 9)&0.5177&0.0019&(18, 13, 9)&0.4784&0.002\\
(10, 9, 8)&0.5617&0.0019&(17, 13, 9)&0.5127&0.0019&(17, 14, 8)&0.4778&0.002\\
(10, 9, 7)&0.5527&0.0019&(17, 16, 7)&0.5124&0.002&(9, 8, 6)&0.474&0.0021\\
(17, 16, 8)&0.5451&0.0019&(17, 13, 10)&0.5101&0.002&(11, 9, 8)&0.467&0.0021\\
(17, 15, 8)&0.5374&0.0019&(17, 12, 10)&0.4976&0.002&(16, 15, 9)&0.4659&0.0021\\
(17, 14, 9)&0.5314&0.0019&(18, 12, 10)&0.4842&0.002&(18, 14, 9)&0.4641&0.0021\\
(10, 8, 7)&0.5198&0.002&(17, 12, 11)&0.4825&0.002&(18, 13, 10)&0.4516&0.0021\\
        \hline
    \end{tabular}
\end{table}

\noindent\textbf{(c)}

比例大于$1.05$和小于$0.95$的三角形如下表所示

\begin{longtable}[H]{|cc|cc|cc|cc|cc|}
    \caption{概率比偏离超出规定比率的三角形($r=\frac{P_a}{P_b}$)}\\
        \hline
        $(x_1,x_2,x_3)$ & $r$ & $(x_1,x_2,x_3)$ & $r$ & $(x_1,x_2,x_3)$ & $r$ & $(x_1,x_2,x_3)$ & $r$ & $(x_1,x_2,x_3)$ & $r$\\
        \hline
        (19, 15, 13)&inf&(17, 16, 4)&0.9382&(17, 14, 3)&0.895&(14, 13, 2)&0.8479&(13, 12, 1)&0.7917\\
(20, 16, 12)&inf&(15, 14, 4)&0.9381&(20, 16, 5)&0.8947&(5, 4, 4)&0.8475&(17, 16, 2)&0.7915\\
(19, 19, 11)&5.0&(6, 5, 5)&0.9379&(7, 4, 4)&0.8942&(16, 14, 2)&0.8453&(5, 4, 3)&0.7912\\
(20, 15, 12)&2.0&(7, 6, 4)&0.937&(16, 14, 3)&0.8938&(20, 19, 8)&0.8451&(6, 6, 1)&0.7909\\
(18, 18, 12)&2.0&(13, 13, 4)&0.9363&(15, 15, 3)&0.8937&(7, 7, 2)&0.8439&(9, 9, 1)&0.7882\\
(19, 16, 12)&1.6&(14, 13, 4)&0.9361&(13, 13, 3)&0.8931&(8, 6, 2)&0.842&(20, 15, 5)&0.7881\\
(20, 20, 4)&1.4&(9, 7, 4)&0.9359&(13, 11, 3)&0.893&(12, 11, 2)&0.8414&(6, 5, 1)&0.7861\\
(18, 15, 13)&1.4&(11, 10, 4)&0.9358&(12, 12, 3)&0.8919&(10, 8, 2)&0.8405&(18, 18, 1)&0.7856\\
(19, 14, 13)&1.333&(11, 11, 4)&0.9351&(17, 16, 3)&0.8919&(9, 9, 2)&0.8402&(20, 16, 11)&0.7833\\
(0, 0, 0)&1.324&(8, 7, 4)&0.9348&(19, 19, 1)&0.8919&(14, 13, 1)&0.8396&(4, 4, 4)&0.782\\
(18, 17, 12)&1.239&(12, 11, 4)&0.9344&(20, 14, 6)&0.8917&(13, 13, 1)&0.8388&(10, 9, 1)&0.7762\\
(20, 10, 10)&1.215&(20, 19, 6)&0.9342&(10, 7, 3)&0.8897&(19, 17, 2)&0.8368&(15, 14, 1)&0.776\\
(19, 17, 12)&1.2&(8, 8, 4)&0.9339&(20, 18, 3)&0.8894&(18, 17, 2)&0.8367&(16, 15, 1)&0.7748\\
(20, 18, 9)&1.198&(16, 15, 4)&0.9335&(8, 8, 3)&0.8893&(10, 9, 2)&0.835&(17, 17, 1)&0.7714\\
(20, 15, 11)&1.193&(19, 19, 4)&0.9334&(10, 10, 3)&0.8892&(8, 7, 2)&0.832&(4, 4, 3)&0.7687\\
(20, 12, 8)&1.181&(19, 17, 4)&0.9332&(11, 11, 3)&0.8885&(16, 15, 2)&0.8317&(5, 5, 1)&0.7687\\
(17, 14, 14)&1.152&(8, 6, 4)&0.9328&(10, 8, 3)&0.8882&(13, 11, 2)&0.8298&(20, 20, 2)&0.7619\\
(16, 16, 13)&1.146&(20, 17, 4)&0.9328&(14, 12, 2)&0.8873&(2, 2, 0)&0.8293&(16, 16, 1)&0.7617\\
(17, 17, 12)&1.144&(16, 13, 4)&0.9327&(6, 6, 3)&0.886&(11, 9, 2)&0.8291&(17, 15, 2)&0.7598\\
(17, 15, 13)&1.14&(10, 8, 4)&0.9325&(9, 7, 3)&0.8836&(8, 5, 3)&0.829&(20, 19, 2)&0.7581\\
(20, 11, 9)&1.138&(16, 16, 4)&0.9324&(11, 10, 3)&0.8831&(13, 12, 2)&0.8282&(4, 4, 2)&0.7472\\
(2, 2, 2)&1.136&(15, 11, 4)&0.9322&(12, 11, 3)&0.8829&(8, 8, 2)&0.8269&(5, 4, 2)&0.7446\\
(17, 16, 13)&1.125&(10, 6, 4)&0.9319&(13, 12, 3)&0.8825&(9, 7, 2)&0.8254&(7, 6, 1)&0.7437\\
(19, 18, 11)&1.121&(12, 10, 4)&0.9318&(9, 9, 3)&0.8814&(6, 6, 2)&0.8229&(4, 3, 2)&0.7405\\
(20, 20, 0)&1.102&(13, 12, 4)&0.9302&(5, 5, 4)&0.8812&(7, 7, 1)&0.8224&(14, 14, 1)&0.7378\\
(19, 13, 13)&1.094&(12, 9, 4)&0.9299&(15, 14, 3)&0.8807&(15, 15, 2)&0.8214&(3, 2, 1)&0.7333\\
(19, 14, 12)&1.093&(15, 13, 4)&0.9289&(18, 18, 3)&0.8805&(16, 15, 14)&0.8214&(5, 5, 2)&0.7305\\
(1, 1, 0)&1.091&(9, 6, 4)&0.9289&(16, 15, 3)&0.8796&(19, 16, 3)&0.8188&(5, 3, 3)&0.7225\\
(20, 18, 10)&1.085&(10, 7, 4)&0.9267&(15, 13, 2)&0.8786&(9, 8, 2)&0.8187&(4, 3, 3)&0.7217\\
(20, 14, 11)&1.075&(7, 7, 4)&0.9267&(8, 7, 3)&0.8783&(10, 10, 2)&0.8183&(19, 19, 2)&0.7192\\
(20, 12, 12)&1.062&(18, 14, 4)&0.9266&(8, 6, 3)&0.8763&(17, 17, 2)&0.8172&(15, 15, 1)&0.7166\\
(20, 17, 11)&1.061&(11, 8, 4)&0.9263&(14, 13, 3)&0.8751&(12, 12, 1)&0.8163&(11, 10, 1)&0.7146\\
(20, 14, 7)&1.056&(10, 5, 5)&0.9248&(6, 4, 4)&0.8744&(11, 11, 2)&0.816&(19, 18, 1)&0.7119\\
        (17, 12, 5)&1.054&(19, 18, 4)&0.9242&(18, 17, 3)&0.8735&(15, 14, 2)&0.8156&(9, 8, 1)&0.707\\
(14, 11, 4)&0.9496&(15, 15, 4)&0.924&(18, 16, 2)&0.8735&(18, 18, 2)&0.814&(6, 3, 3)&0.7053\\
(9, 9, 4)&0.9493&(19, 15, 4)&0.9236&(9, 8, 3)&0.8728&(19, 18, 2)&0.8134&(2, 2, 1)&0.7\\
(12, 9, 3)&0.9493&(17, 17, 4)&0.9232&(11, 9, 3)&0.8723&(11, 10, 2)&0.8127&(20, 17, 3)&0.7\\
        (4, 4, 0)&0.9492&(15, 12, 4)&0.9203&(15, 13, 3)&0.8715&(5, 5, 3)&0.8126&(11, 11, 1)&0.6879\\
(16, 14, 4)&0.9469&(18, 15, 3)&0.92&(7, 7, 3)&0.8695&(13, 13, 2)&0.8126&(6, 4, 2)&0.6813\\
(19, 16, 4)&0.9466&(3, 3, 0)&0.9199&(16, 16, 3)&0.8677&(6, 4, 3)&0.8124&(3, 3, 2)&0.6797\\
(19, 19, 10)&0.9459&(17, 13, 4)&0.9193&(19, 19, 3)&0.8669&(7, 4, 3)&0.8098&(5, 3, 2)&0.6667\\
(16, 11, 5)&0.9455&(6, 6, 4)&0.9178&(7, 6, 3)&0.8668&(8, 7, 1)&0.8096&(20, 19, 9)&0.6667\\
(9, 8, 4)&0.945&(7, 5, 4)&0.9154&(9, 6, 3)&0.8654&(12, 12, 2)&0.8089&(3, 3, 3)&0.6585\\
(13, 10, 4)&0.9438&(6, 5, 4)&0.9127&(17, 17, 3)&0.8653&(7, 5, 2)&0.8085&(4, 3, 1)&0.6512\\
(12, 12, 4)&0.9436&(20, 20, 5)&0.9091&(19, 18, 3)&0.865&(13, 10, 3)&0.8085&(20, 19, 1)&0.6471\\
(18, 16, 4)&0.9432&(8, 5, 4)&0.9087&(7, 5, 3)&0.8634&(12, 11, 1)&0.8073&(5, 4, 1)&0.6452\\
(17, 15, 4)&0.9432&(15, 12, 3)&0.9037&(11, 8, 3)&0.8631&(14, 14, 2)&0.8018&(3, 2, 2)&0.6407\\
(18, 17, 4)&0.942&(14, 7, 7)&0.9036&(14, 12, 3)&0.863&(20, 18, 2)&0.8&(20, 14, 12)&0.6364\\
(18, 18, 4)&0.9418&(12, 10, 3)&0.9034&(20, 19, 3)&0.8575&(17, 16, 1)&0.7993&(4, 4, 1)&0.6299\\
(18, 15, 4)&0.941&(20, 18, 4)&0.899&(14, 14, 3)&0.8536&(18, 17, 1)&0.798&(20, 20, 3)&0.6286\\
(10, 10, 4)&0.9409&(19, 17, 3)&0.898&(19, 15, 12)&0.8532&(7, 6, 2)&0.7947&(20, 20, 1)&0.6\\
(13, 11, 4)&0.9407&(10, 9, 3)&0.8978&(8, 8, 1)&0.8522&(6, 5, 2)&0.7946&(3, 3, 1)&0.5893\\
(10, 9, 4)&0.9404&(16, 13, 3)&0.8968&(14, 11, 3)&0.8517&(10, 10, 1)&0.7938&(15, 15, 15)&0.5\\
(11, 9, 4)&0.9389&(17, 15, 3)&0.8966&(6, 5, 3)&0.8502&(12, 10, 2)&0.7933&(4, 2, 2)&0.4107\\
(14, 14, 4)&0.9383&(18, 16, 3)&0.8955&(8, 4, 4)&0.8496&(16, 16, 2)&0.793&(20, 19, 10)&0.0\\
(14, 14, 4)&0.9383&(18, 16, 3)&0.8955&(8, 4, 4)&0.8496&(16, 16, 2)&0.793&(20, 19, 10)&0.0\\
        \hline
\end{longtable}

发现$r=\frac{P_a}{P_b}>1.05$的数目较少,

\section{蒙卡的应用-一维谐振子}

\noindent\textbf{(a)}

由于积分为高斯收敛,通过实验我们选用积分区间为$[-10,10]^7$即可保证有限体积积分与无穷积分之间差距不大。

\begin{table}[H]
    \centering
    \caption{蒙卡积分结果与理论值对比}
    \begin{tabular}{|c|ccccc|}
        \hline
        $x$ & 0.00 & 0.25 & 0.50 & 0.75 & 1.00 \\
        \hline
        PathInt & 0.07927(24) & 0.07430(11) & 0.061744(91) & 0.045271(64) & 0.029288(41) \\
        \cline{0-0}
        $e^{-\frac{x^2}{2}}/\pi^\frac{1}{4}$ & 0.076355 & 0.071729 & 0.059465 & 0.043506 & 0.028089 \\
        \hline
        $x$ & 1.25 & 1.50 & 1.75 &2.00 &~\\
        \hline
        PathInt& 0.016811(24) & 0.008486(12) & 0.0037856(54) & 0.0014933(22) &~\\
        \cline{0-0}
        $e^{-\frac{x^2}{2}}/\pi^\frac{1}{4}$& 0.016005 & 0.0080477 & 0.0035712 & 0.0013985 &~\\
        \hline
    \end{tabular}
\end{table}

其中括号内为误差位,如$0.07927(24)$意为$0.07927\pm 0.00024$

\begin{figure}[H]
    \centering
    \includegraphics[width=0.6 \textwidth]{2a.png}
    \caption{路径积分与理论值散点图}
\end{figure}

\noindent\textbf{(b)(c)}

\begin{figure}[H]
    \centering
    \subfigure[$N_{conf}=25$]{
    \includegraphics[width=0.4 \textwidth]{G1.png}
    }
    \subfigure[$N_{conf}=100$]{
    \includegraphics[width=0.4 \textwidth]{G2.png}
    }

    \subfigure[$N_{conf}=400$]{
    \includegraphics[width=0.4 \textwidth]{G3.png}
    }
    \subfigure[$N_{conf}=10000$]{
    \includegraphics[width=0.4 \textwidth]{G4.png}
    }
    \caption{关联函数图像}
\end{figure}

误差采用Jackknife方法计算,可以发现随着组态数目增加,误差不断降低.

\begin{table}[H]
    \centering
    \begin{tabular}{|c|c|c|c|c|}
        \hline
        $n$ & $N_{conf}=25$ & $N_{conf}=100$ & $N_{conf}=400$ & $N_{conf}=1000$ \\
        \hline
        0&0.549048&0.509441&0.477543&0.486245\\
1&0.352685&0.319708&0.291602&0.29905\\
2&0.24339&0.20487&0.174041&0.184478\\
3&0.186428&0.133964&0.103921&0.115638\\
4&0.134949&0.0840271&0.062857&0.0697869\\
5&0.103729&0.0554977&0.0384455&0.0413718\\
6&0.109807&0.0396117&0.0314812&0.0241943\\
7&0.0989372&0.0427743&0.0238644&0.0147024\\
8&0.0986716&0.0389618&0.0174467&0.00979733\\
9&0.106408&0.0403206&0.0157832&0.00771997\\
10&0.0887247&0.0350935&0.0173943&0.00757189\\
11&0.106408&0.0403206&0.0157832&0.00771997\\
12&0.0986716&0.0389618&0.0174467&0.00979733\\
13&0.0989372&0.0427743&0.0238644&0.0147024\\
14&0.109807&0.0396117&0.0314812&0.0241943\\
15&0.103729&0.0554977&0.0384455&0.0413718\\
16&0.134949&0.0840271&0.062857&0.0697869\\
17&0.186428&0.133964&0.103921&0.115638\\
18&0.24339&0.20487&0.174041&0.184478\\
19&0.352685&0.319708&0.291602&0.29905\\
\hline
    \end{tabular}
    \caption{不同组态数目计算得到的关联函数值}
\end{table}

\begin{figure}[H]
    \centering
    \subfigure[$N_{conf}=25$]{
    \includegraphics[width=0.4 \textwidth]{Npar=25.png}
    }
    \subfigure[$N_{conf}=100$]{
    \includegraphics[width=0.4 \textwidth]{Npar=100.png}
    }

    \subfigure[$N_{conf}=400$]{
    \includegraphics[width=0.4 \textwidth]{Npar=400.png}
    }
    \subfigure[$N_{conf}=10000$]{
    \includegraphics[width=0.4 \textwidth]{Npar=1000.png}
    }
    \caption{$\Delta E_n$图像}
\end{figure}

实际发现其图像相当不稳定,故而接下来使用更多的组态。

\begin{figure}[H]
    \centering
    \includegraphics[width=0.6 \textwidth]{2cc.png}
    \caption{10000个组态平均得到的$\Delta E_n$}
\end{figure}

此时图像稳定了很多,误差更小,平台大致出现在$0.5$处.

\end{document}