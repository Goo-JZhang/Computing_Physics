%!TEX program = pdflatex
\documentclass[UTF8]{ctexart}
\usepackage{amsmath}
\usepackage{amssymb}
\usepackage{yhmath}
\usepackage{esvect}
\usepackage{arcs}
\usepackage{geometry}
\usepackage{graphicx}
\usepackage{subfigure}
\usepackage{float}
\usepackage{MnSymbol}
\usepackage{diagbox}
\usepackage{multirow}
\usepackage{makecell}
\usepackage{longtable}
\usepackage[justification=centering]{caption}
\geometry{left=2.0cm,right=2.0cm,top=2cm,bottom=2cm}
\newcommand*{\dif}{\mathop{}\!\mathrm{d}}
\newcommand*{\C}{^{\circ}\mathrm{C}}
\pagestyle{plain}

\begin{document}

\title{计算物理作业2}
\author{姓名:张健霖\\学号: \texttt{1900011327}}
\maketitle

\section{多项式插值}

\noindent\textbf{(a)}

\begin{figure}[H]
\centering
\subfigure[函数图像]{\includegraphics[width=0.3 \textwidth]{1-a-2}}
\subfigure[差值]{\includegraphics[width=0.3 \textwidth]{1-a-1}}
\caption{插值多项式与sin图像,均匀取点}
\end{figure}

此处使用Lagrange内插法,近似找到最大偏差为$\sigma_1=\max(|P_n(x)-f(x)|)\approx 2.1\times 10^{-4}$

理论估计最大偏差上界为
$$\sigma_t=\frac{|x_n-x_0|^{n+1}}{(n+1)!}\max_{\eta\in [x_0,x_n]}|f^{n+1}(\eta)|=\frac{\pi^5}{32\cdot 5!}\approx 0.08$$

\noindent\textbf{(b)}

支撑点选为$x_n=[0,0.1,1.4,1.5,\frac{\pi}{2}]$
\begin{figure}[H]
\centering
\subfigure[函数图像]{\includegraphics[width=0.3 \textwidth]{1-b-1}}
\subfigure[差值]{\includegraphics[width=0.3 \textwidth]{1-b-2}}
\caption{插值多项式与sin图像,非均匀取点}
\end{figure}

近似找到最大偏差为$\sigma_2=1.1\times 10^{-3}$,依然小于$\sigma_t$

\noindent\textbf{(c)}

$$\sigma_t(n)=\frac{\pi^{n}}{2^{n}n!}<10^{-8}$$

要求$n>13$,我们有$\sigma_t(13)\approx 5.7\times 10^{-8},\ \sigma_t(14)\approx 6.4\times 10^{-9}$

\begin{figure}[H]
\centering
\subfigure[n=10偏差]{\includegraphics[width=0.4 \textwidth]{1-c-1}}
\subfigure[n=14偏差]{\includegraphics[width=0.4 \textwidth]{1-c-2}}
\caption{不同支撑点数目误差,均匀取点}
\end{figure}

采用Neville内插,因Lagrange内插在点数较多的情况下失真严重,均匀取点在n=14时已经接近机器精度,误差曲线长相奇怪很正常.

\section{Runge效应}

\noindent\textbf{(a)}

\begin{figure}[H]
\centering
\subfigure[$f(x),P(x)$图像]{\includegraphics[width=0.4 \textwidth]{2-a-1}}
\subfigure[误差曲线]{\includegraphics[width=0.4 \textwidth]{2-a-2}}
\caption{均匀取点多项式Runge拟合图像,n=41表示取[-1,1]中均匀41个点作图}
\end{figure}

\begin{table}[H]
    \caption{均匀取点多项式Runge拟合数据}
    \begin{tabular}{|c|ccccccccc|}
        \hline
        $x$ & -1.00 & -0.95 & -0.90 & -0.85 & -0.80 & -0.75 & -0.70 & -0.65 & -0.60 \\
        $f(x)$ & 0.0385 & 0.0424 & 0.0471 & 0.0525 & 0.0588 & 0.0664 & 0.0755 & 0.0865 & 0.100 \\
        $P(x)$ & 0.0385 & -40.0 & 0.0471 & 3.45 & 0.0588 & -0.447 & 0.0755 & 0.202 & 0.100 \\
        $|P(x)-f(x)|$ & 2.4e-10 & 4.0e1 & 1.5e-11 & 3.4e0 & 9.8e-12 & 5.1e-1 & 5.1e-13 & 1.2e-1 & 2.2e-15 \\
        \hline
        $x$ & -0.55 & -0.50 & -0.45 & -0.40 & -0.35 & -0.30 & -0.25 & -0.20 & -0.15 \\
        $f(x)$ & 0.117 & 0.138 & 0.165 & 0.200 & 0.246 & 0.308 & 0.390 & 0.500 & 0.640 \\
        $P(x)$ & 0.0807 & 0.138 & 0.180 & 0.200 & 0.238 & 0.308 & 0.395 & 0.500 & 0.637 \\
        $|P(x)-f(x)|$ & 3.6e-2 & 7.9e-15 & 1.5e-2 & 4.6e-15 & 7.7e-3 & 2.3e-15 & 4.8e-3 & 4.4e-16 & 3.2e-3 \\
        \hline
        $x$ & -0.1 & -0.05 &  0.00 & 0.05 & 0.10 & 0.15 & 0.20 & 0.25 & 0.30 \\
        $f(x)$ & 0.800 & 0.941 & 1.00 & 0.941 & 0.800 & 0.640 & 0.500 & 0.390 & 0.308 \\
        $P(x)$ & 0.800 & 0.942 & 1.00 & 0.942 & 0.800 & 0.637 & 0.500 & 0.395 & 0.308 \\
        $|P(x)-f(x)|$ & 0.0e0 & 1.3e-3 & 0.0e0 & 1.3e-3 & 1.1e-16 & 3.2e-3 & 2.2e-16 & 4.8e-3 & 1.7e-16 \\
        \hline
        $x$ & 0.35 & 0.40 & 0.45 & 0.50 & 0.55 & 0.60 & 0.65 & 0.70 & 0.75 \\
        $f(x)$ & 0.246 & 0.200 & 0.165 & 0.138 & 0.117 & 0.100 & 0.0865 & 0.0755 & 0.0664 \\
        $P(x)$ & 0.238 & 0.200 & 0.180 & 0.138 & 0.0807 & 0.100 & 0.202 & 0.0755 & -0.447\\
        $|P(x)-f(x)|$ & 7.7e-3 & 1.9e-15 & 1.5e-2 & 7.1e-14 & 3.6e-2 & 3.3e-13 & 1.2e-1 & 3.2e-12 & 5.1e-1 \\
        \hline
        $x$ & 0.80 & 0.85 & 0.90 & 0.95 & 1.00 & ~ & ~ & ~ & ~ \\
        $f(x)$ & 0.0588 & 0.0525 & 0.0471 & 0.0424 & 0.0385 & ~ & ~ & ~ & ~\\
        $P(x)$ & 0.0588 & 3.45 & 0.0471 & -40.0 & 0.0385 & ~ & ~ & ~ & ~\\
        $|P(x)-f(x)|$ & 1.1e-11 & 3.4e0 & 3.3e-11 & 4.0e1 & 1.2e-10 & ~ & ~ & ~ & ~ \\
        \hline
    \end{tabular}
\end{table}

\noindent\textbf{(b)}

\begin{figure}[H]
\centering
\subfigure[$f(x),C(x)$图像]{\includegraphics[width=0.4 \textwidth]{2-b-1}}
\subfigure[误差曲线]{\includegraphics[width=0.4 \textwidth]{2-b-2}}
\caption{Chebyshev多项式Runge拟合图像,n=41表示取$\{\cos(\frac{\pi i}{40}),i=0,1...40\}$中的41个点作图}
\end{figure}

\begin{table}[H]
    \caption{Chebyshev多项式Runge拟合数据}
    \begin{tabular}{|c|ccccccccc|}
        \hline
        $x$ & 1.00 & 0.997 & 0.988 & 0.972 & 0.951 & 0.924 & 0.891 & 0.853 & 0.809 \\
        $f(x)$ & 0.0385 & 0.0387 & 0.0394 & 0.0406 & 0.0424 & 0.0448 & 0.0480 & 0.0522 & 0.0576 \\
        $C(x)$ & 0.0370 & 0.0387 & 0.0409 & 0.0406 & 0.0408 & 0.0448 & 0.0498 & 0.0522 & 0.0554 \\
        $|C(x)-f(x)|$ & 1.4e-3 & 2.1e-12 & 1.5e-3 & 1.6e-13 & 1.6e-3 & 3.1e-13 & 1.8e-3 & 1.4e-13 & 2.2e-3 \\
        \hline
        $x$ & 0.760 & 0.707 & 0.649 & 0.588 & 0.522 & 0.454 & 0.383 & 0.309 & 0.233 \\
        $f(x)$ & 0.117 & 0.138 & 0.165 & 0.200 & 0.246 & 0.308 & 0.390 & 0.500 & 0.640 \\
        $C(x)$ & 0.0647 & 0.0769 & 0.0866 & 0.0999 & 0.128 & 0.169 & 0.215 & 0.284 & 0.423 \\
        $|C(x)-f(x)|$ & 3.5e-14 & 2.8e-3 & 5.1e-15 & 3.9e-3 & 9.4e-16 & 6.1e-3 & 1.4e-16 & 1.1e-2 & 6.1e-16 \\
        \hline
        $x$ & 0.156 & 0.0785 & 0.00 & -0.785 & -0.156 & -0.233 & -0.309 & -0.383 & -0.454 \\
        $f(x)$ & 0.620 & 0.867 & 1.00 & 0.867 & 0.620 & 0.423 & 0.295 & 0.215 & 0.163\\
        $C(x)$ & 0.644 & 0.867 & 0.962 & 0.867 & 0.644 & 0.423 & 0.284 & 0.215 & 0.169 \\
        $|C(x)-f(x)|$ & 2.3e-2 & 4.4e-16 & 3.8e-2 & 0.0e0 & 2.3e-2 & 9.4e-16 & 1.1e-2 & 1.9e-16 & 6.1e-3 \\
        \hline
        $x$ & -0.522 & -0.588 & -0.649 & -0.707 & -0.760 & -0.809 & -0.853 & -0.891 & -0.924 \\
        $f(x)$ & 0.128 & 0.104 & 0.0866 & 0.0741 & 0.0647 & 0.0576 & 0.0522 & 0.0480 & 0.0448\\
        $C(x)$ & 0.128 & 0.0999 & 0.0866 & 0.0769 & 0.0647 & 0.0554 & 0.0522 & 0.0498 & 0.0448\\
        $|C(x)-f(x)|$ & 8.0e-16 & 3.9e-3 & 8.1e-15 & 2.8e-3 & 4.5e-15 & 2.2e-3 & 2.0e-13 & 1.8e-3 & 9.1e-13 \\
        \hline
        $x$ & -0.951 & -0.972 & -0.988 & -0.997 & -1.00 & ~ & ~ & ~ & ~ \\
        $f(x)$ & 0.0424 & 0.0406 & 0.0394 & 0.0387 & 0.0385 & ~ & ~ & ~ & ~\\
        $C(x)$ & 0.0408 & 0.0406 & 0.0408 & 0.0387 & 0.0370 & ~ & ~ & ~ & ~\\
        $|C(x)-f(x)|$ & 1.6e-3 & 5.3-e13 & 1.5e-3 & 7.1e-13 & 1.4-e3 & ~ & ~ & ~ & ~ \\
        \hline
    \end{tabular}
\end{table}

在误差范围内可以认为在$x_k=\cos\frac{\pi(k+\frac{1}{2})}{20}$上$f(x_k)=C(x_k)$,其中$C(x)$为Chebyshev拟合多项式.

对比\noindent\textbf{(a)}可知$C(x)$与$f(x)$的差距至多为0.038,且在靠近$x=0$处偏离较多,在$|x|\rightarrow 1$处拟合情况较好;
而$P_{20}(x)$与$f(x)$的差距最大可达40,在靠近$x=0$处拟合较好,在远离$x=0$处偏离相当大.\ 综合来看,相比$P_{20}(x)$,$C(x)$拟合效果更好.

\noindent\textbf{(c)}

\begin{figure}[H]
\centering
\subfigure[$f(x),S(x)$图像]{\includegraphics[width=0.4 \textwidth]{2-c-1}}
\subfigure[误差曲线]{\includegraphics[width=0.4 \textwidth]{2-c-2}}
\caption{三次样条函数Runge拟合图像,边界条件为$S''|_{boundary}=0$,表示41个均匀取点值}
\end{figure}

\begin{figure}[H]
\centering
\subfigure[周期边界条件]{\includegraphics[width=0.4 \textwidth]{2-c-3}}
\subfigure[$S'|_{boundary}=f'|_{boundary}$]{\includegraphics[width=0.4 \textwidth]{2-c-4}}
\caption{其他边界条件误差曲线,表示41个均匀取点值}
\end{figure}

\begin{table}[H]
    \caption{三次样条函数Runge拟合数据}
    \begin{tabular}{|c|ccccccccc|}
        \hline
        $x$ & -1.00 & -0.95 & -0.90 & -0.85 & -0.80 & -0.75 & -0.70 & -0.65 & -0.60 \\
        $f(x)$ & 0.0385 & 0.0424 & 0.0471 & 0.0525 & 0.0588 & 0.0664 & 0.0755 & 0.0865 & 0.100 \\
        $S(x)$ & 0.0385 & 0.0425 & 0.0471 & 0.0524 & 0.0588 & 0.0664 & 0.0755 & 0.0865 & 0.100 \\
        $|S(x)-f(x)|$ & 8.3e-17 & 9.4e-5 & 5.6e-17 & 2.8e-5 & 1.2e-16 & 4.0e-6 & 6.9e-17 & 1.3e-5 & 1.4e-16 \\
        \hline
        $x$ & -0.55 & -0.50 & -0.45 & -0.40 & -0.35 & -0.30 & -0.25 & -0.20 & -0.15 \\
        $f(x)$ & 0.117 & 0.138 & 0.165 & 0.200 & 0.246 & 0.308 & 0.390 & 0.500 & 0.640 \\
        $S(x)$ & 0.117 & 0.138 & 0.165 & 0.200 & 0.246 & 0.308 & 0.389 & 0.500 & 0.643 \\
        $|S(x)-f(x)|$ & 1.4e-6 & 5.0e-16 & 8.4e-5 & 2.2e-16 & 1.1e-4 & 5.6e-17 & 8.2e-4 & 4.4e-16 & 3.2e-3 \\
        \hline
        $x$ & -0.1 & -0.05 &  0.00 & 0.05 & 0.10 & 0.15 & 0.20 & 0.25 & 0.30 \\
        $f(x)$ & 0.800 & 0.941 & 1.00 & 0.941 & 0.800 & 0.640 & 0.500 & 0.390 & 0.308 \\
        $S(x)$ & 0.800 & 0.939 & 1.00 & 0.939 & 0.800 & 0.643 & 0.500 & 0.389 & 0.308 \\
        $|S(x)-f(x)|$ & 0.0e0 & 2.3e-3 & 0.0e0 & 2.3e-3 & 0.0e0 & 3.2e-3 & 1.1e-16 & 8.2e-4 & 1.7e-16 \\
        \hline
        $x$ & 0.35 & 0.40 & 0.45 & 0.50 & 0.55 & 0.60 & 0.65 & 0.70 & 0.75 \\
        $f(x)$ & 0.246 & 0.200 & 0.165 & 0.138 & 0.117 & 0.100 & 0.0865 & 0.0755 & 0.0664 \\
        $S(x)$ & 0.246 & 0.200 & 0.165 & 0.138 & 0.117 & 0.100 & 0.0864 & 0.0755 & 0.0664\\
        $|S(x)-f(x)|$ & 1.1e-4 & 1.7e-16 & 8.4e-5 & 5.6e-17 & 1.4e-6 & 4.4e-16 & 1.3e-5 & 2.1e-16 & 4.0e-6 \\
        \hline
        $x$ & 0.80 & 0.85 & 0.90 & 0.95 & 1.00 & ~ & ~ & ~ & ~ \\
        $f(x)$ & 0.0588 & 0.0525 & 0.0471 & 0.0424 & 0.0385 & ~ & ~ & ~ & ~\\
        $S(x)$ & 0.0588 & 0.0524 & 0.0471 & 0.0425 & 0.0385 & ~ & ~ & ~ & ~\\
        $|S(x)-f(x)|$ & 3.5e-16 & 2.8e-5 & 2.1e-16 & 9.4e-5 & 2.8e-17 & ~ & ~ & ~ & ~ \\
        \hline
    \end{tabular}
\end{table}

三次样条函数$S_\Delta$拟合的效果相比$P(x)$和$C(x)$更好,在误差范围内$S(x_n)=f(x_n),\ x_n=-1,-0.9...0.9,1$是成立的,在其余点偏差最大不超过0.0032.

表格使用的样条函数边界条件为$S''|_{boundary}=0$

\section{样条函数在计算机绘图中的应用}

\noindent\textbf{(a)}

\begin{table}[H]
    \centering
    \begin{tabular}{|c|ccccccccc|}
        \hline
        $\phi$ & $0$ & $\frac{\pi}{4}$ & $\frac{\pi}{2}$ & $\frac{3\pi}{4}$ & $\pi$ & $\frac{5\pi}{4}$ & $\frac{3\pi}{2}$ & $\frac{7\pi}{4}$ & $2\pi$ \\
        \hline
        $x(\phi)$ & 0.000 & 0.207 & 0.000 & -1.207 & -2.000 & -1.207 & 0.000 & 0.207 & 0.000\\
        $y(\phi)$ & 0.000 & 0.207 & 1.000 & 1.207 & 0.000 & -1.207 & -1.000 & -0.207 & 0.000\\
        \hline
    \end{tabular}
    \caption{心形线支撑点数据表,本表只展示小数点后三位}
\end{table}

\noindent\textbf{(b)}

考虑周期边界条件的内插

\begin{table}[H]
    \centering
    \begin{tabular}{|c|c|c|}
        \hline
        \multirow{2}{*}{$t\in [0,1)$} & $S_\Delta(X;t)$ & $0.425417x^2-0.218311x^3$\\
        ~ & $S_\Delta(Y;t)$ & $0.033612x^2+0.173495x^3$ \\
        \hline
        \multirow{2}{*}{$t\in [1,2)$} & $S_\Delta(X;t)$ & $-0.044815+0.134446x+0.290971x^2-0.173495x^3$\\
        ~ & $S_\Delta(Y;t)$ & $0.455185-1.331942x+1.365554x^2-0.281689x^3$ \\
        \hline
        \multirow{2}{*}{$t\in [2,3)$} & $S_\Delta(X;t)$ & $-4.044815+6.134446x-2.709029x^2+0.326505x^3$\\
        ~ & $S_\Delta(Y;t)$ & $-0.051845-0.571398x+0.985281x^2-0.218311x^3$ \\
        \hline
        \multirow{2}{*}{$t\in [3,4)$} & $S_\Delta(X;t)$ & $-2.834797+4.924428x-2.305690x^2+0.281689x^3$\\
        ~ & $S_\Delta(Y;t)$ & $-14.761863+14.138621x-3.918058x^2+0.326505x^3$ \\
        \hline
        \multirow{2}{*}{$t\in [4,5)$} & $S_\Delta(X;t)$ & $33.221439-22.117749x+4.454855x^2-0.281689x^3$\\
        ~ & $S_\Delta(Y;t)$ & $-14.761863+14.138621x-3.918058x^2+0.326505x^3$ \\
        \hline
        \multirow{2}{*}{$t\in [5,6)$} & $S_\Delta(X;t)$ & $38.823376-25.478911x+5.127087x^2-0.326505x^3$\\
        ~ & $S_\Delta(Y;t)$ & $53.340074-26.722542x+4.254174x^2-0.218311x^3$ \\
        \hline
        \multirow{2}{*}{$t\in [6,7)$} & $S_\Delta(X;t)$ & $-69.176624+28.521089x-3.872913x^2+0.173495x^3$\\
        ~ & $S_\Delta(Y;t)$ & $67.029871-33.567440x+5.394991x^2-0.281689x^3$ \\
        \hline
        \multirow{2}{*}{$t\in [7,8]$} & $S_\Delta(X;t)$ & $-84.548340+35.108967x-4.814038x^2+0.218311x^3$\\
        ~ & $S_\Delta(Y;t)$ & $-89.098413+33.344682x-4.163884x^2+0.173495x^3$ \\
        \hline
    \end{tabular}
    \caption{三次样条函数表,各系数保留至小数点后第六位}
\end{table}

\noindent\textbf{(c)}

\begin{figure}[H]
\centering
\subfigure[拟合曲线与严格曲线图像]{\includegraphics[width=0.4 \textwidth]{3-c-1}}
\subfigure[误差曲线]{\includegraphics[width=0.4 \textwidth]{3-c-2}}
\caption{三次样条函数心形线,边界条件为$S''|_{boundary}=0$}
\end{figure}

\begin{figure}[H]
\centering
\subfigure[拟合曲线与严格曲线图像]{\includegraphics[width=0.4 \textwidth]{3-c-3}}
\subfigure[误差曲线]{\includegraphics[width=0.4 \textwidth]{3-c-4}}
\caption{三次样条函数心形线,周期边界条件}
\end{figure}

\begin{figure}[H]
\centering
\subfigure[拟合曲线与严格曲线图像]{\includegraphics[width=0.4 \textwidth]{3-c-5}}
\subfigure[误差曲线]{\includegraphics[width=0.4 \textwidth]{3-c-6}}
\caption{三次样条函数心形线,边界条件为$S'|_{boundary}=f'|_{boundary}$}
\end{figure}

误差曲线定义为
$$(\delta_x(t),\delta_y(t))=\Big(S_{\Delta}(X;t)-x(\frac{t\pi}{4}),S_{\Delta}(Y;t)-y(\frac{t\pi}{4})\Big)$$

其中$x(\phi)=(1-\cos\phi)\cos\phi,\ y(\phi)=(1-\cos\phi)\sin\phi$,误差曲线只表征随着参数变化,
$(S_\Delta(X;t),S_\Delta(Y;t))$与$(x(\frac{t\pi}{4}),y(\frac{t\pi}{4}))$的差,并不能表征实际曲线的拟合程度.

从图中可以发现,在周期边界条件和一阶导边界条件下的曲线拟合得更好.

\noindent\textbf{(d)}

我们知道

样条函数在支撑点上满足$$S_\Delta(t_n-0)'=S_\Delta(t_n+0)',\ \ S_\Delta(t_n-0)''=S_\Delta(t_n+0)''$$

从而曲线$\Big(S_\Delta(X;t),S_\Delta(Y,t)\Big)$在$t$的切向量
$$\Big(S_\Delta(X;t)',S_\Delta(Y,t)'\Big)$$在支撑点处连续

且曲线的曲率$$\kappa=\frac{S_\Delta(X;t)'S_\Delta(Y,t)''-S_\Delta(X;t)''S_\Delta(Y,t)'}{\Big(S_\Delta(X,t)'^2+S_\Delta(Y,t)'^2\Big)^\frac{3}{2}}$$
也在支撑点处连续

从而三次样条函数画出来的曲线可以平滑连接所有点.

\section{计算积分}

\noindent\textbf{(a)}

$$\int_{-\infty}^\infty \exp(-x^2)\cos(x)\dif x$$

\begin{table}[H]
    \centering
    \begin{tabular}{|c|c|c|c|c|}
        \hline
        \diagbox{步长}{积分区间} & $[-1,1]$ & $[-2,-2]$ & $[-5,-5]$ & $[-10,-10]$\\
        \hline
        $(\frac{1}{10})^0$ & 0.39753222069282595 & 1.984756009634227 & 0.79457574227332 & 1.9695117250368583 \\
        $(\frac{1}{10})^1$ & 1.3076329469329262 & 1.3853171833541256 & 1.380388447041953 & 1.380388447043143 \\
        $(\frac{1}{10})^2$ & 1.3123006286642913 & 1.38522383499722 & 1.3803884470421255 & 1.3803884470431436 \\
        $(\frac{1}{10})^3$ & 1.3123482531240573 & 1.3852229124828541 & 1.3803884470421273 & 1.3803884470431431 \\
        $(\frac{1}{10})^4$ & 1.3123487207481235 & 1.3852229033434258 & 1.3803884470421275 & 1.380388447043143 \\
        $(\frac{1}{10})^5$ & 1.3123487254158732 & 1.3852229032521146 & 1.3803884470421268 & 1.380388447043143 \\
        $(\frac{1}{10})^6$ & 1.3123487254625423 & 1.3852229032512016 & 1.3803884470421262 & 1.3803884470431433 \\
        $(\frac{1}{10})^7$ & 1.3123487254630086 & 1.3852229032511938 & 1.3803884470421244 & 1.3803884470431391\\
        \hline
    \end{tabular}
    \caption{梯形法计算积分}
\end{table}

\begin{table}[H]
    \centering
    \begin{tabular}{|c|c|c|c|c|}
        \hline
        \diagbox{$N$}{积分区间} & $[-1,1]$ & $[-2,2]$ & $[-5,-]$ & $[-10,10]$\\
        \hline
        $1$ & 0.39753222069282595 & -0.03048798073154619 & 0.00000000000039395 & 0 \\
        $2$ & 1.4658440735642753 & 2.6565040064228183 & 6.6666666666799 & 13.333333333333334 \\
        $3$ & 1.3005398372242634 & 1.0939699168715564 & 1.322335495072119 & 2.6666666667226955 \\
        $4$ & 1.3126450137840004 & 1.4022029681600827 & 1.0057721065687437 & 1.5267464641332582 \\
        $5$ & 1.3123464394811315 & 1.3849903873608702 & 1.3989044317152834 & 1.0001879003212255 \\
        $6$ & 1.3123487307035782 & 1.385223210481481 & 1.3808320403994059 & 1.399308030162181 \\
        $7$ & 1.3123487254595367 & 1.3852229055118186 & 1.3803748389878259 & 1.3808275150342368 \\
        $8$ & 1.312348725463015 & 1.3852229032481522 & 1.380388506822054 & 1.3803748113613006 \\
        $9$ & 1.3123487254630134 & 1.3852229032511931 & 1.380388446982128 & 1.380388507032048 \\
        $10$ & 1.3123487254630135 & 1.3852229032511927 & 1.3803884470421426 & 1.3803884469829149 \\
        $11$ & 1.3123487254630135 & 1.3852229032511922 & 1.3803884470421268 & 1.3803884470431578 \\
        $12$ & 1.312348725463014 & 1.3852229032511925 & 1.3803884470421268 & 1.380388447043143 \\
        \hline
    \end{tabular}
    \caption{外推法计算积分}
\end{table}

其中$N$表示按$h^2_i=|\frac{(b-a)}{2^i}|^2,i=0,1...N-1$的序列作多项式内插.

精确值为
$$\sqrt{\frac{\pi}{\sqrt{e}}}=1.380388447043143$$

理论分析可知,在积分区域取为$[-L,L]$时,与精确值的误差为
$$\sigma_1\sim \exp(-L^2)$$

在$L$取到5时,有$\exp(-L^2)=1.4\mathrm{e}-11$,两种方法精度均能达到1e-12,理论预言与实际接近.

梯形法的理论误差随步长的关系为
$$\sigma\sim h^3N\sim h^2,\ \ Nh=b-a$$

从表中也能得出这一点.

龙贝格积分在$[-10,10]$区间上,$N=12$时与真实值的距离小于1e-15,已经相当接近了,而在$[-10,10]$区间上截断误差为$\sigma\sim 4e-44$忽略不计.

\noindent\textbf{(b)}

对于$\exp(-x^2)$的权函数,采用厄米多项式$$H_n(x)$$

积分满足
$$\int_{-\infty}^\infty \exp(-x^2)f(x)\dif x\approxeq\sum_{i=1}^{n}w_if(x_i)$$

其中$x_1,...,x_n$为$H_n$的零点

故而有
$$\int_{-\infty}^\infty \exp(-x^2)\cos(x)\dif x\approxeq\sum_{i=1}^n w_i\cos(x_i)$$

查表可得不同权重系数

\begin{table}[H]
    \centering
    \begin{tabular}{|c|c|c|c|c|c|c|c|}
    \hline
    \multicolumn{2}{|c|}{N=5} & \multicolumn{2}{c}{N=10} & \multicolumn{2}{|c|}{N=15} & \multicolumn{2}{|c|}{N=20} \\
    \hline
    $x$          & $w$          & $x$           & $w$          & $x$           & $w$          & $x$           & $w$          \\
    \hline
    -2.020183 & 0.01995324 & -3.436159 & 7.640433e-06    & -4.499991 & 1.522476e-09    & -5.387481  & 2.229394e-13    \\
    -0.9585725& 0.3936193 & -2.532732 & 0.001343646 & -3.669950 & 1.059116e-06    & -4.603682  & 4.399341e-10    \\
    0           & 0.9453087  & -1.756684 & 0.03387439 & -2.967167 & 1.000044e-04    & -3.944764  & 1.086069e-07    \\
    0.9585725 & 0.3936193 & -1.036611 & 0.2401386 & -2.325732 & 0.002778069 & -3.347855 & 7.802556e-06    \\
    2.020183  & 0.01995324 & -0.3429013 & 0.6108626 & -1.719993 & 0.03078003 & -2.788806 & 2.283386e-04    \\
                &             & 0.3429013  & 0.6108626 & -1.136116 & 0.1584890 & -2.254974 & 0.003243773 \\
                &             & 1.036611  & 0.2401386 & -0.565070 & 0.4120287 & -1.738538 & 0.02481052 \\
                &             & 1.756684  & 0.03387439 & 0            & 0.5641003 & -1.234076 & 0.1090172 \\
                &             & 2.532732  & 0.001343646 & 0.565070  & 0.4120287 & -0.7374737 & 0.2866755 \\
                &             & 3.436159  & 7.640433e-06    & 1.136116  & 0.1584890 & -0.2453407 & 0.4622437  \\
                &             &              &             & 1.719993  & 0.03078003 & 0.2453407  & 0.4622437  \\
                &             &              &             & 2.325732  & 0.002778069 & 0.7374737  & 0.2866755 \\
                &             &              &             & 2.967168  & 1.000044e-04    & 1.234076  & 0.1090172 \\
                &             &              &             & 3.669950  & 1.059116e-06    & 1.738538  & 0.02481052 \\
                &             &              &             & 4.499991  & 1.522476e-09    & 2.254974  & 0.003243773 \\
                &             &              &             &              &             & 2.788806  & 2.283386e-04    \\
                &             &              &             &              &             & 3.347855  & 7.802556e-06    \\
                &             &              &             &              &             & 3.944764   & 1.086069e-07    \\
                &             &              &             &              &             & 4.603682   & 4.399341e-10    \\
                &             &              &             &              &             & 5.387481   & 2.229394e-13    \\
    \hline
    \end{tabular}
    \caption{高斯节点权重表,保留七位有效数字}
\end{table}

\begin{table}[H]
    \centering
    \begin{tabular}{|c|c|c|c|c|}
        \hline
        $N$ & 5 & 10 & 15 & 20 \\
        \hline
        积分结果 & 1.380390075935656 & 1.3803884470431405 & 1.3803884470431418 & 1.3803884470431431 \\
        \hline
    \end{tabular}
\end{table}

在取节点数为$20$时,积分结果精度为1e-16.

\end{document}