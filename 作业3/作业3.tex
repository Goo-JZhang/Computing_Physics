%!TEX program = pdflatex
\documentclass[UTF8]{ctexart}
\usepackage{amsmath}
\usepackage{amssymb}
\usepackage{yhmath}
\usepackage{esvect}
\usepackage{arcs}
\usepackage{geometry}
\usepackage{graphicx}
\usepackage{subfigure}
\usepackage{float}
\usepackage{MnSymbol}
\usepackage{diagbox}
\usepackage{multirow}
\usepackage{makecell}
\usepackage{longtable}
\usepackage[justification=centering]{caption}
\geometry{left=2.0cm,right=2.0cm,top=2cm,bottom=2cm}
\newcommand*{\dif}{\mathop{}\!\mathrm{d}}
\newcommand*{\C}{^{\circ}\mathrm{C}}
\pagestyle{plain}

\begin{document}

\title{计算物理作业3}
\author{姓名:张健霖\\学号: \texttt{1900011327}}
\maketitle

\section{Householder 与 Givens 在QR分解中的比较}

\noindent\textbf{(a)}

对于Householder方法进行QR分解,对于形如
\begin{equation}\label{1-a-1}
A_k=\left(
    \begin{array}{cc}
        R_{k} & C_{k,n-k}\\
        0_{n-k,k} & M_{n-k}
    \end{array}
\right)
\end{equation}

其中$R_{k}$为$k\times k$的上三角矩阵

选用
\begin{equation}\label{1-a-2}
    \begin{aligned}
        P_k&=1_{n,n}-2\frac{vv^T}{||v||^2}\\
        v&=x_{n-k}-||x_{n-k}||e_{k+1}
    \end{aligned}
\end{equation}

其中$x_{n-k}$的前$k$个分量为0,后$n-k$个分量为$M_{n-k}$的第一列

可得
\begin{equation}\label{1-a-3}
    P_k\left(
        \begin{array}{cc}
            R_{k} & C_{k,n-k}\\
            0_{n-k,k} & M_{n-k}
        \end{array}
    \right)
    =\left(
        \begin{array}{ccc}
            R_{k} & c_{k} & C_{k,n-k-1}'\\
            0 & ||x_{n-k}|| & b_{k}^T\\
            0_{n-k-1,k} & 0_{n-k-1} & M_{n-k-1}
        \end{array}
    \right)
    =\left(
        \begin{array}{cc}
            R_{k+1} & C_{k+1,n-k-1}\\
            0_{n-k-1,k+1} & M_{n-k-1}
        \end{array}
    \right)
\end{equation}

采用如下运算顺序
\begin{equation}\label{1-a-4}
    \begin{aligned}
        v'=\frac{v}{||v||}\\
        P_kA_k=A_k-(2v')(v'^TA_k)
    \end{aligned}
\end{equation}

$v=x_{n-k}-||x_{n-k}||e_{k+1}$只需计算$||x_{n-k}||$,再对第$k+1$分量进行操作即可,共进行了$n-k$次浮点数乘法和$n-k$次加法运算,以及1次开方运算.

其中$||v||$共进行了$n$次浮点乘法运算和$n-1$次浮点加法运算和1次开方运算,$v'=\frac{v}{||v||}$进行了$n-k$次乘法运算,
$v'^TA_k$共进行了$(n-k)^2$次浮点乘法和$(n-k)(n-k-1)$次浮点加法,$2v'$进行了$n-k$次浮点乘法运算,
$(2v')(v'^TA)$共进行了$(n-k)^2$次浮点乘法运算,再有$A_k-2v'v'^TA_k$有$(n-k)^2$次浮点加法运算.

从而$P_kA_k$共进行了$4(n-k)^2$次运算,进而$P_{n-1}P_{n-2}...P_0A=R$要进行$\frac{4}{3}n^3$次运算;
而对$Q=P_0P_1...P_{n-1}$,定义$Q_k=P_0P_1..P_k$,$Q_{k+1}=Q_kP_{k+1}=Q_k-2Q_kv'v'^T$,其中$Q_kv'$需要$nk$次乘法和$n(k-1)$次加法,$2v$需要$k$次乘法,
$(Q_kv')(2v'^T)$需要$nk$次乘法,$Q_k-(Q_kv'v^T)$需要$nk$次加法,从而$Q$的计算需要$2n^3$次运算.

故而使用Householder方法得到$QR$分解需要$\frac{10}{3}n^3$次运算.

~\

对于Givens方法,获得Givens矩阵的归一系数需要2次浮点数乘法,1次加法,1次开方运算,得到特值参量$c,s$还需2次乘法,共6次运算.

利用Givens矩阵得到QR的过程如下

\begin{equation}\label{1-a-5}
    \left(
        \begin{array}{cc}
            a_1 & A_{n-1,n-1}\\
            a_{n,1} & b_n^T \\
        \end{array}
    \right)
    \stackrel{G_{n,1}}{\longrightarrow}
    \left(
        \begin{array}{cc}
            a_1' & A_{n-1,n-1}'\\
            0 & b_n'^T \\
        \end{array}
    \right)
    =
    \left(
        \begin{array}{cc}
            a_1'' & A_{n-2,n-1}'\\
            a_{n-1,1}' & b_{n-1}'^T\\
            0 & b_n'^T \\
        \end{array}
    \right)
    \stackrel{G_{n-1,1}}{\longrightarrow}
    \left(
        \begin{array}{cc}
            a_1''' & A_{n-2,n-1}''\\
            0 & b_{n-1}''^T\\
            0 & b_n''^T \\
        \end{array}
    \right)
\end{equation}

逐步将第1列除了第一个元素外均化为0,共进行了$n-1$次Givens变换,每次Givens变换均做了$4n$次浮点数乘法和$2n$次浮点数加法.\ 
随后对第2列到第$n-1$列进行处理,对于第$k$列,共进行$n-k$次Givens变换,每次需要$6(n-k+1)$次运算,
故而变换本身运算次数为$\sum_{k=1}^{n-1}6(n-k)(n-k+1)\approx 2n^3$.

考虑到我们需要获得$\sum_{k=1}^{n-1}(n-k)\approx \frac{n^2}{2}$个Givens变换矩阵,每个矩阵系数需要6次运算,总共为$3n^2$次运算.

获得$Q=G_{n,1}G_{n-1,1}...G_{1,1}G_{n-1,2}...G_{2,2}...G_{n-1,n-1}$,其中每个乘法需要$6n$次计算,总共需要约$3n^3$次运算.

从而总运算量为$5n^3$次.

~\

\noindent\textbf{(b)(c)}

见Eigen.py中的HouseholderQRD()与GivensQRD()函数

~\

\noindent\textbf{(d)}

\begin{table}[H]
    \centering
    \caption{Householder与Givens时间对比(10次平均)}
    \begin{tabular}{|c|ccccccc|}
        \hline
        矩阵编号 & 0 & 1 & 2 & 3 & 4 & 5 & 6 \\
        \hline
        Householder用时$/\mathrm{ms}$ & 1.203942 & 1.099992 & 2.100587 & 1.099420 & 1.099968 & 1.200032 & 1.100087 \\
        Givens用时$/\mathrm{ms}$ & 0.8002520 & 0.8956194 & 0.8000135 & 0.8038998 & 0.9010792 & 0.7951260 & 0.8999109 \\
        \hline
        矩阵编号 & 7 & 8 & 9 & 10 & 11 & 12 & 13 \\
        \hline
        Householder用时$/\mathrm{ms}$ & 1.099920 & 1.196051 & 1.200032 & 1.099992 & 1.903915 & 1.196051 & 1.100492 \\
        Givens用时$/\mathrm{ms}$ & 0.8039713 & 0.9005547 & 0.7954836 & 0.9999514 & 0.8000135 & 0.9001970 & 0.8037090 \\
        矩阵编号 & 14 & 15 & 16 & 17 & 18 & 19 & ~ \\
        \hline
        Householder用时$/\mathrm{ms}$ & 1.103401 & 1.200461 & 1.095891 & 1.203823 & 1.099896 & 1.100206 & ~ \\
        Givens用时$/\mathrm{ms}$ & 0.8000374 & 0.8999586 & 0.8001566 & 0.7998466 & 0.8960962 & 0.7999659 & ~ \\
        \hline
    \end{tabular}
\end{table}

对比得知在$n=6$时Givens相比Householder更快一点.

第一个随机矩阵为

\begin{equation}\label{1-d-1}
    A=\left(
        \begin{array}{cccccc}
            0.09762701 & 0.43037873 & 0.20552675 & 0.08976637 & -0.1526904  & 0.29178823\\
            -0.12482558 &  0.783546  &  0.92732552 & -0.23311696 & 0.58345008 & 0.05778984\\
            0.13608912 & 0.85119328 & -0.85792788 & -0.8257414 & -0.95956321 & 0.66523969\\
            0.5563135 &  0.7400243 &  0.95723668 & 0.59831713 & -0.07704128 & 0.56105835\\
            -0.76345115 & 0.27984204 & -0.71329343 & 0.88933783 & 0.04369664 & -0.17067612\\
            -0.47088878 & 0.54846738 & -0.08769934 & 0.1368679 & -0.9624204 &  0.23527099
        \end{array}
    \right)
\end{equation}

Householder对$A$进行QR分解的结果为

\begin{equation}\label{1-d-2}
    Q_H=\left(
        \begin{array}{cccccc}
            0.09073395 & 0.2749425  & 0.0222482  & 0.1120602 & -0.01358722 & 0.95022991\\
            -0.11601214 & 0.50052942 & 0.56822308 & -0.48811337 & 0.40973828 & -0.08362947\\
             0.12648041 & 0.54376939 &-0.77091154 &-0.26333566 & 0.10268563 & -0.11884001\\
             0.51703443 & 0.47279441 & 0.23857814 & 0.61179186 &-0.08732801 & -0.26515251\\
            -0.70954691 & 0.1786921 & -0.11901777 & 0.54603841 & 0.38818615 &-0.04000813\\
            -0.4376412  & 0.35032321 & 0.10601031 &-0.08599391 &-0.81429167 &-0.06355917
        \end{array}
    \right)
\end{equation}

\begin{equation}\label{1-d-3}
    R_H=\left(
        \begin{array}{cccccc}
            1.07596994 & -0.000166549168 & 0.841976472 & -0.450826248 & 0.147449561 & 0.412135485\\
            0 & 1.56539584 & 0.348539971 & -0.0512675673 & -0.637502887 & 0.788075148\\
            0 & 0 & 1.49686018 & 0.557516065 & 0.942263532 & -0.294400771\\
            0 & 0 & 0 & 1.18118226 & 0.0102760210 & 0.0591318647\\
            0 & 0 & 0 & 0 & 0.949984321 & -0.218804747\\
            0 & 0 & 0 & 0 & 0 & 0.0364846482
        \end{array}
    \right)
\end{equation}

Givens变换对$A$进行QR分解的结果为

\begin{equation}\label{1-d-4}
    Q_G=\left(
        \begin{array}{cccccc}
            0.09073395 & 0.2749425  & 0.0222482  & 0.1120602 & -0.01358722 & -0.95022991\\
            -0.11601214 & 0.50052942 & 0.56822308 & -0.48811337 & 0.40973828 & 0.08362947\\
             0.12648041 & 0.54376939 &-0.77091154 &-0.26333566 & 0.10268563 & 0.11884001\\
             0.51703443 & 0.47279441 & 0.23857814 & 0.61179186 &-0.08732801 & 0.26515251\\
            -0.70954691 & 0.1786921 & -0.11901777 & 0.54603841 & 0.38818615 & 0.04000813\\
            -0.4376412  & 0.35032321 & 0.10601031 &-0.08599391 &-0.81429167 & 0.06355917
        \end{array}
    \right)
\end{equation}

\begin{equation}\label{1-d-5}
    R_G=\left(
        \begin{array}{cccccc}
            1.07596994 & -0.000166549168 & 0.841976472 & -0.450826248 & 0.147449561 & 0.412135485\\
            0 & 1.56539584 & 0.348539971 & -0.0512675673 & -0.637502887 & 0.788075148\\
            0 & 0 & 1.49686018 & 0.557516065 & 0.942263532 & -0.294400771\\
            0 & 0 & 0 & 1.18118226 & 0.0102760210 & 0.0591318647\\
            0 & 0 & 0 & 0 & 0.949984321 & -0.218804747\\
            0 & 0 & 0 & 0 & 0 & -0.0364846482
        \end{array}
    \right)
\end{equation}

\section{幂次法求矩阵最大模的本征值和本征矢}

\noindent\textbf{(a)}

\begin{equation}\label{2-a-1}
    -\omega^2x(t)=-Ax(t)\rightarrow x\exp(-\mathrm{i}\omega t)=Ax\exp(-\mathrm{i}\omega t)
\end{equation}

得到

\begin{equation}\label{2-a-2}
    Ax=\omega^2x
\end{equation}

\noindent\textbf{(b)}

解得本征值

\begin{equation}\label{2-b-1}
    \lambda=\omega_{max}^2=4.000000000000001
\end{equation}

本征向量(保留14位有效数字)
\begin{equation}\label{2-b-2}
    q=\left(
        \begin{array}{c}
            0.31622776601693 \\
            -0.31622776601691	\\
            0.31622776601687	\\
            -0.31622776601681\\	
            0.31622776601677	\\
            -0.31622776601675	\\
            0.31622776601677	\\
            -0.31622776601681\\	
            0.31622776601687	\\
            -0.31622776601691	\\
        \end{array}
    \right)
\end{equation}

理论值为

\begin{equation}\label{2-b-3}
    \hat{\lambda}_{max}=4
\end{equation}

\begin{equation}\label{2-b-4}
    \hat{q}=\frac{1}{\sqrt{10}}(1,-1,1,-1,1,-1,1,-1,1,-1)^T
\end{equation}

且理论上本征值满足

\begin{equation}\label{2-b-5}
    \lambda(i)=2(\cos\frac{2i\pi}{5}-\cos\frac{i\pi}{5}),\ i=0,1,...,9
\end{equation}

可知仅当$i=5$时取到最大本征值$\hat{\lambda}_{max}=4$,无简并.

\section{关联函数的拟合与数据分析}

\noindent\textbf{(a)}

\begin{figure}[H]
\centering
\includegraphics[width=0.45 \textwidth]{H3-3a1}
\caption{$C$的模与实部相对误差图像}
\end{figure}

从图中可以发现两者相差并不多,并且从题意知虚部是舍入误差导致的,故而我们选用$C$的实部误差作为对$C$误差的估计.

\begin{table}[H]
    \centering
    \caption{相对误差表,因$C(t)=C(48-t)$故而只列出前25个时刻}
    \begin{tabular}{|c|ccccccccc|}
        \hline
        $t$ & 0 & 1 & 2 & 3 & 4 & 5 & 6 & 7 & 8 \\
        \cline{0-0}
        $\frac{\Delta C}{C}\times 100\%$ & 0.229 & 0.230 & 0.240 & 0.246 & 0.253 & 0.266 & 0.271 & 0.279 & 0.293 \\
        \hline
        $t$ & 9 & 10 & 11 & 12 & 13 & 14 & 15 & 16 & 17 \\
        \cline{0-0}
        $\frac{\Delta C}{C}\times 100\%$ & 0.304 & 0.324 & 0.344 & 0.368 & 0.393 & 0.413 & 0.445 & 0.471 & 0.494 \\
        \hline
        $t$ & 18 & 19 & 20 & 21 & 22 & 23 & 24 & ~ & ~\\
        \cline{0-0}
        $\frac{\Delta C}{C}\times 100\%$ & 0.520 & 0.546 & 0.581 & 0.614 & 0.652 & 0.690 & 0.720 & ~ & ~ \\
        \hline
    \end{tabular}
\end{table}

$\frac{\Delta C(t)}{C(t)}$作为函数的映射关系由表给出,对于非整数的$t$可用线性内插方法得到.

~\

\noindent\textbf{(b)(c)}

\begin{figure}[H]
    \centering
    \includegraphics[width=0.7 \textwidth]{H3-3b1}
    \caption{$m_{eff}$及其误差图像}
\end{figure}

\begin{figure}[H]
    \centering
    \includegraphics[width=0.5 \textwidth]{H3-3b2}
    \caption{$m_{eff}$及其误差图像放大图}
\end{figure}

扫描得到$[t_{min},t_{max}]=[16,20]$,$m=1.28782996$,$A=0.00999145$,$\frac{\chi^2}{d}_{min}=3.28791e-5$

对于$m$的误差,采用$\chi^2(A,m)=\frac{1}{t_{max}-t_{min}-1}\sum_{t=t_{min}}^{t_{max}}\Big(\frac{C(t)-C_{fit}(t)}{\Delta C(t)}\Big)$,
自由度$d=3$,$99.7\%$置信度要求$\Delta \chi^2=14.2$,在极小值$A_0,m_0$附近求解$\chi^2(A_n,m_{n+1})=\Delta\chi^2,\chi^2(A_{n+1},m_{n+1})=\Delta\chi^2$,
当$n\rightarrow\infty$时即可得到$\sigma_{m}=m_\infty-m_0$,考虑到在$\sigma_{m}$较大时$\chi^2$等值面可能偏离对称,故而需要在两侧搜索取$\sigma_{m}=\max(|m_\infty-m_0|,|m_{-\infty}-m_0|)$,
实际程序中使用$n=50$次迭代.

$99.7\%$置信度下$\sigma_m=0.00085505$

\noindent\textbf{(d)}

\begin{equation}\label{3-d-1}
    C(t)=\sum_{n=0} A_n\Big(e^{-E_nt}+e^{E_n(t-N_t)}\Big)
\end{equation}

\begin{equation}\label{3-d-2}
    C(\Delta+\frac{N_t}{2})=\sum_{n=0}^\infty A_ne^{-E_n\frac{N_t}{2}}\Big(e^{-E_n\Delta}+e^{E_n\Delta}\Big)
\end{equation}

在$\Delta$比较小的时候,高阶噪声可以忽略,从而

\begin{equation}\label{3-d-3}
    R(t)=R(\Delta+\frac{N_t}{2})=\frac{e^{-m\Delta}+e^{m\Delta}}{e^{-m}e^{-m\Delta}+e^me^{m\Delta}}
\end{equation}

可以解得

\begin{equation}\label{3-d-4}
    m\Delta=\frac{1}{2}\ln\frac{1-R(t)e^{-m}}{e^{m}R(t)-1}
\end{equation}

同理
\begin{equation}\label{3-d-5}
    R(t-1)=R(\Delta-1+\frac{N_t}{2})=\frac{e^me^{-m\Delta}+e^{-m}e^{m\Delta}}{e^{-m\Delta}+e^{m\Delta}}
\end{equation}

解得

\begin{equation}\label{3-d-6}
    m\Delta=\frac{1}{2}\ln\frac{e^{m}-R(t-1)}{R(t-1)-e^{-m}}
\end{equation}

于是有

\begin{equation}\label{3-d-7}
    (e^{4m}-1)R(t)-(1+R(t)R(t-1))(e^{3m}-e^{m})=0
\end{equation}

\begin{equation}\label{3-d-8}
    R(t)e^{2m}-(1+R(t)R(t-1))e^{m}+R(t)=0
\end{equation}

于是有
\begin{equation}\label{3-d-9}
    m_{eff}(t)=\ln\frac{1+R(t)R(t-1)+\sqrt{[1+R(t)R(t-1)]^2-4R(t)^2}}{2R(t)}
\end{equation}

\begin{figure}[H]
    \centering
    \includegraphics[width=0.6 \textwidth]{H3-3d1}
    \caption{修正后的$m_{eff}(t)$与未修正对比,其中fine meff为修正后的数据点}
\end{figure}

发现修正后的$m_{eff}(t)$确实在靠近$\frac{N_t}{2}$处有平台.

\end{document}